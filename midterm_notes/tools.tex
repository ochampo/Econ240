
\documentclass{report}
\usepackage{a4wide}

\begin{document}

\linespread{1.0}\selectfont

\textbf{Study first exam}\\


























the six componets of the macor environment 
\begin{enumerate}
   
\item politcal factors - 
this includes like fiscal polices  tariffs the politcal climate 
fdereral vanking system some politcal policies affect certain types of indurtroies 
\item economic condtions
the health of the nation so like unempolyment rate, ex
exchange rate, growth, trade. so pretty much how the e
econmy is doing  
\item sociocultural forces 
how the people feel about the attiude of the social we
well being 
\item technological factors 
\item environmental forces 
\item legal and regulatory 


\end{enumerate}

\textbf{comptetive pressures  created by rivalry among competing sellers} 


\begin{enumerate}
\item rivalry increases when buyers demand is growing slowly or declining.
so when ever there is somthing going when intrest is going up they have sales or promation the same affect happens when the intrest is going out of Bussiness 
\item rivalry increases as it becomes less costly for buyers to switch brands 
an example would be the phone company where there is a cost to switch to another one but with team mobile they cut that cost 
\item rivalry increases as the product of rivals sellers becone less strongly differentiated. 

so if they are same product they competition will be higher because the buyer will go to the one the is cheaper. 
\item rivalry is more intense when there is excess in supply or unused production capcity especually if the industy's product has hig fixed cost or high storage cost 
so when there is a lot of something and they have to get rid of because it cost them money to keep. 
\item rivalry intensifes as the number of competitors increase and they become more equal in size and capabilty. 

when companyies cut the prices 
\item rivalry becomes more intense as the diversity of competitors increases in terms  of long term direction objectives and countries of orgin a diverse 

\item rivalry is stronger when exit barrier keep unprofitale firms from leaving the industry 



\end{enumerate}

\textbf{the choice of competive weapons}
\begin{itemize}
\item sales promtions 
\item heavey advertising 
\item rebates 
\item low interest rate 
The people might say there product is better because if this and that. 

\end{itemize}

\large{\textbf{competitive  pressure associated with the threat of new entrants}}

\begin{itemize}
\item expected reaction of incument firms to new entry and what are  known as barries to entry
\textbf{barries are entry are high under the following condtions}
\begin{itemize}
\item industries incumbent enjoy large cost advantages over potenial entrants 
\begin{enumerate}


\item scale  economies in production, distibution,advertising, other activites 
\item the learning-based cost saving that accrue from experince in perfoming ceartain activietes sucn as manufacturing or new prodict development of inventory mangement
\item cost sacing  accruing from patent or proretary technology 
\item favorable locations and 
\item low fixed cost becasue incunbent have older facilties taht been mostly deprciate the 

 
\end{enumerate}

\end{itemize}

\item costumers have strong brand preference and high degree of loalty to seller 
\item patent and other forms of intellectual property protectin are in place
some case patents that are used that prevent to entry but once they are up they better watch out boom!
\item thare are string "network effects " in customer demand. 
when everyone is using a certian device you cannot compete with them, so like xbox and playstation 
\item capital requirements are high 
\item there are fiffeiculties in building a network of distributors/dealers or in securing adequate space on retailers' shelves 
\item there are restrictive reulatory polices 

\item there are restrive trade polices 


\end{itemize}


{\huge\textbf{competitive pressures from the sellers of substitute products from the sellers of substitute prodicts}}

\begin{enumerate}
\item good substitutes are readily available and attractively priced 
\item buyer view the subtitutes as comparable or better in trms of qualit perfromance and other relavent attributes 
\item the cost that buyers incur in switching to the substitutes are 
\end{enumerate}

{\huge\textbf{competitive pressures stemming from supplier bargaining power}}

it all depends on the bargaing power to influence the terms and conditions of supply in their favor. 
\textbf{the powers are stronger when}
\begin{itemize}
\item demand for suppliers products is high and the prdoucts are in short supply 
\item suppliers provide differentaited inputs that enhance the preformance of the industry  prodcts 
\item industry members are incapable of intergrating backward to self manufacture items they have been buying  form suppliers 
\item suppliers provide an item that accounts for no more than a a small fraction of the costs of the indutry product
\item good subtitutes are not available for the sppliers  prodcuts 
\item industry members are not major customers of suppliers
page 58 

\end{itemize}

{\huge\textbf{Competive pressures stemming from buyer  bargaing power and price sensitivity page 60} }\\

whether buyers are to exert strong competitive pressures on industty members depend on 

buyer bargaining power is stronger whe 
\begin{itemize}
\item buyers demand is weak in relation to industry supply
\item industry goods are standardixed or dfferentiation is weak 
\item buyers costs of switching to competing brands or substiutes are relativly low 
\item  buyers are large and few in numbers  relative to the number of sellers toh larger the buyers the mo
\item buyers pose a credible threat of integrating bakward into business of sellers 
\item buyers are wll infromed about sllers products prices and cost 
\item buyers have discretion to delay purches or perhaps even not made a purches at all 
\item buyers price sensitivity increases when buyers are earning low profits or have low income
\item buyers are more price sensitve if the product represent a large fraction of their total purchase 

\end{itemize}

{\huge\textbf{is the collective strength of the competitive forces conducive to good profitablity page 62}}
Intense competitive pressures from just one of the five forces may suffice to destroy the conditions for good  profitability \\

also  ther strongest competitive forces determine the extent of the competive pressure on industry profitability 1


{\huge\textbf{matching company startegy to competitve conditions page 63 }}
\begin{enumerate}
\item pursuing avenues that shield the firm from as many of the different competive pressures as possible 
\item initatiing actions calculated to shift the competive forces in the company's favor by altering the underlying factors driving  the five forces 
\end{enumerate}

{\huge\textbf{complementors and the value net page 63}}

Complementors  are the producers of complemetary products,which are product that enhance the calue of the focal firms products when they are use 


{\huge\textbf{industry dynamics and the forces driving change}}


Industry and competive conditions change because forces are enticing or pressuring certain industy particioants(competitors,customers suppliers complemetors  to alter their actions  in important ways  the most powerful of the change agents are called driving forces becuse they have the biggest influence in reshaping  the industry  landscape and altering competive conditions 
\textbf{driving forces has 3 steps  }
\begin{enumerate}
\item identifyin what the driving forces are 
\item assesing whether the drivers if change are  on the whole acting to make the industry more or less attractive 
\item determining what strategy  changer are needed to prepare for the impact of the driving forces. all three steos merit futher discussion 
\end{enumerate}


{\huge\textbf{identifying the forces driving industry change page 65}}
\begin{itemize}
\item changes in an industry long term growth rate 
\item increasing globlization 
\item emerging new internet cabilites and applications 
\item shifts in buyer demograpics 
\item shifts in buy demograpics 

\item technological change and manufacturing porcess innovation 
\item product innovation 
\item entry it exit of major firms 
\item diffusion of technical know how across companies and countries 
\item changes in cost effieciency 
\item reduction in uncertainty and business risk 
\item regulatory influences  and government policy changes 
\item Changin scietal concerns attitude and lifestyles 

\end{itemize}

There many factors but only like three or four can make an impact 


{\huge\textbf{assesing the impact of the forces driving industy change 67}}
\begin{itemize}
\item changes in long rerm industy growth rate 
\item increasing glblization 
\item emerging new internet cabailites and applications shifts in buyers demographics 
\item technological change and mnufacturing process innovation 
\item producti and marketing innovation entry or exit of major firms 
\item diffusion of technical know how across compies and countrie 
\item reduction unertainty and business ris  
\item regulaory influences and goverbnebt policy change 
\item changing social concerns attitudes and lifestyles 

\end{itemize}

{\huge\textbf{using startegic groups tp asess the market postitons of key competitors page 67}}

A strategic  group consists of those industy member with simialr competive approaches and postions in the market 

the procedure for constructing a strategic group map is straightfoward 


\begin{itemize}
\item the competive characteristics that delineate strtegic aooriaches used in the industy typical variable used in creating strategic group map are price/quality rang hig mdeduin low geographic coveage local regin=onal natiln golo=bal o.... see page 68
\item plot the firms on a two variable map using pairs of thes variables 
\item  assign firms occupying  about the same map locatuin to the same strategic group 
\item draw circle around each stratigic group making the circles proprional to the size of the group share of total industy sales  revenus 


\end{itemize}



{\huge\textbf{the value of stategic group maps pg 70}}

\begin{itemize}
\item prevailing competive pressire from the industy five force may caus the progit potential  of differen strategic groups to vary
\item industry driving force may favor some strategig groups and hurt other 



\end{itemize}

all in page 71

\textbf{current startegy}
they need to know competitors stategy to succed 

\textbf{objective}
they need to have stratigic objective as well 
\textbf{resources and capabilities }
what ever you have at hand 

\textbf{assumption}
making decisotoin based on your situation 



\end{document}



