\documentclass{article}
\newcommand\tab[1][1cm]{\hspace*{#1}}
\begin{document}

\title{econ 240 }
\maketitle
\tab[4cm]{\huge chapter 7}\\ \\ \\
 
 




\textbf{\huge business-government relations  }

\begin{itemize}
\item understanding why sometimes governments and business collaborate and other times work are's length frome each other 

\item defining public  policy and the elements of the public  process 
\item eplaining the reasons for reulation
\item knowing the major types of government regulation of business 
\item comparing the cost and benefits ofregulation for business and society 
\item examing the conditions that affect the reulation of business in global context  

\end{itemize}




\textbf{\huge notes for intro  }


\begin{itemize}
\item the automakers need helo so the government gave them a 256 billion dollar loan because they wanted workers to stay working they also owend some of the cars stock(one third of gm stocks) 
\item the greek government did was on spending spree 
\item the government asked banks for help selling bonds to them or something like that 

\end{itemize}

{\huge how busuness and government relate}

\textbf{intro}\\
according to the book understating the givernment authority and its relationship with business is essential for manager in developing  their strategies and achieving  their orgainization goals

\textbf{\huge business-government relations  }

\begin{itemize}
\item government can work and they will both come out better off 

\end{itemize}

\textbf{\huge Working in arms lengths  }

\textbf{ why do businessses sometimes welcome government regulation and involvment in the private sector and other times oppose it?}

\begin{itemize}
\item companies often prefer to operate without government constraints which can e costly or restrict innovation 
\item they can also set a requirment that everyone must follow 

\end{itemize} 
The business - government relationship is one that requires  managers to keep a careful eye tranined towards signifcant forces that might alter this relatiinship or promote forces that may encourage postive busuness government relationship 

\textbf{\huge legitimacy issues   }
\begin{itemize}
\item economic sactions: human rights violation  or lack of legitimate government will cause other government to move from that country examples  
\begin{enumerate}
\item quwait
\item lebanon 
\item libya 
\item qatar 
\item saudi arabia 
\item syria
\item united arab emirates 
\item Republic of Yemen
\end{enumerate}

\tab[3cm]{ government's public policy roles }

\textbf{\huge intro }
 
\textit{publc policy} : is the plan of action undertaken by government officals to achieve some broad purpose affecting a substantial segment of a nations's citizens \\
in other words to do and what not to do
\begin{itemize}
\item the government does not act unless a huge part of the public is affectedm,the the govenment is working on people benefit 
\item the basic power to make public policy comes from a nation's politcal system 
\end{itemize} 

\textbf{\huge Elements of public policy}\\
the actions of government in any nation can be understood  in terms of sevral basic elements of public policy thes are inputs goals tools and effects \\
\textit{public policy inputs} : are external pressure  that shape a governments's policy  decisions  and startegies to adress problems. 


\begin{enumerate}
\item economic and foreign policy conerns 
\item domestic politcal pressure from constuents and intrest gropus 
\item technical information 
\item media attention 
\end{enumerate}
these are the factors that play a role in shaping national politcal decisions 

\textit{public policy effects} are the outcomes arigning from government regulation \\
there was argument that phones cost more to take take off than leave off 
so it cost  billion to leave them on. but 25 billion in benefit lost meaning that is would cost 23 billion to take them off 

 

\textbf{\huge types of public policy }\\
 created by the government, that have to major types 
 \begin{enumerate}
 \item econimic 
 \item social  
 \end{enumerate}
somtimes they are the diffrent and sometimes they are similar \\

\textit{\huge ecomic polices }\\

\textsc{fical policy} : refers to patterns of government  collecting and spending funt that are intetended to stiumlate or support the economy . \\
\textsc{monetary policy}: refers to polices that affect the supply demand and value of a nations currency \\

example of fiscal policy would be bailouts where they give loans or finaces to troubled comapanys to keep them running \\
monetary policy - lowering intrest rate so more comapnies can gain confidence and barrow 

\begin{enumerate}
\item fical policy referst to patterns of government collecting and spenig funds that are intented to stimulate or support the econmy 
\item monetary policy refers to polices that affect the upply demand and value of a nations currency 
\item taxation policy  rasiing or lowering taxes on busuness or inivduaks 
\item industial policy  driceting economic resources towards the development of specific indutstires 
\item trade policu encouringin or discouraging trade with other countries 
\end{enumerate}


\end{itemize}
\tab[4cm]\textit{\huge social assiantance polices  }

just asstaince for the well being of people 

\textbf{\huge Market failures}

\textit{market failures}: that is the market place fails to adjust  prices for the true cost  of a firms behavior.

\textbf{\huge Negative exernalites}
  
 \textit{Negative exenalites}: or the sillover effect results  when the manufacture or sitribution of a product fives rise to unplanned or intentended cost \\
\textbf{\huge Natural monopolies}\\
 \textit{Natural monopolies}: ex:the elecric company can set the prices  to whatever they want so there would be government intervention \\
  \textbf{\huge ethical argumetns }\\
  \textit{ethical arguments }: utilarian thical argument  in support of safe working condtions\\ 
  
  \textbf{\huge type of regulations }\\ 
   \textit{type of regulations}:
   \begin{enumerate}
   \item economic regulations : aim to modify the normal operation of the free market and the forces of supply and demand 
   \item antitrust a special kind of econimic regulation 
   \begin{itemize}
   \item antitrust laws : one important  kind of econmic regulation occurs when government act to preserce competition in the market place thereby protecting consumers.
   \begin{itemize}
   \item prohibits unfair, anticompetive  practices by business( other counries use the competition law)
 
   \end{itemize}
   \item predatory pricing the practice of the selling below cost to drive rivals out of business 
  \end{itemize} 
  
  if found guilty of the antiturst there will be fine depending on what it is. pay for the damges done to other companies 
  
    
   
   \end{enumerate}
   
 the goverment might impose some remedies 
 \begin{itemize}
 \item \textit{structural remedy}: may require a break up of monopolistic firm
 \item \textit{conduct remedy}: when the the firm will change its conduct 
 often over government supervison 
 \item \textit{intellectual property  remedy}: it involves disclosures  of information to competitors all these are part  of the regulator arsenal 
 \end{itemize}
     
   \tab[4cm]\textbf{social regulations}:
   
   \textit{social regulations}: are amied at such important social goals as prtexting consumers and the enviroment and providing workers with sage  and health wrking conditions
   \begin{enumerate}
   \item equal employment 
   \item employment opportunuty 
   \item protection of pension  benfits 
   \item health care for citizens 
\end{enumerate}    
   
   
   
 {\large dood frank act}: the role of the dodd frank act in regulating a complex finacial product called derivativetes is explored in the disscussion case at the end of this chapter 
 
 
 \tab[4cm]\textbf{the effect of regulation}\\
\textit{the cost and benefits of regulation}\\
there is always a cost to regulation but we hope the benefits is greater than the cost \\

the test of \textit{ cost benefits analysis}  helps the public understand what is at stake when new regulation is sought.

\tab[4cm]\textbf{continous regulatiory reform}

\textit{dergulation} is the removal or sacling down of regulatory  authrity  and regulartory  activities of government.\\ 
\textit{regulation}:  is the increase or expansion of government regulation especuially in areas where the reggulatory activities had previoyky been reduced.



\end{document}